% CVPR 2022 Paper Template
% based on the CVPR template provided by Ming-Ming Cheng (https://github.com/MCG-NKU/CVPR_Template)
% modified and extended by Stefan Roth (stefan.roth@NOSPAMtu-darmstadt.de)

\documentclass[10pt,twocolumn,letterpaper]{article}

%%%%%%%%% PAPER TYPE  - PLEASE UPDATE FOR FINAL VERSION
%\usepackage[review]{cvpr}      % To produce the REVIEW version
\usepackage{cvpr}              % To produce the CAMERA-READY version
%\usepackage[pagenumbers]{cvpr} % To force page numbers, e.g. for an arXiv version

% Include other packages here, before hyperref.
\usepackage{graphicx}
\usepackage{amsmath}
\usepackage{amssymb}
\usepackage{booktabs}

\newcommand{\latex}{\LaTeX\xspace}
\newcommand{\tex}{\TeX\xspace}


% It is strongly recommended to use hyperref, especially for the review version.
% hyperref with option pagebackref eases the reviewers' job.
% Please disable hyperref *only* if you encounter grave issues, e.g. with the
% file validation for the camera-ready version.
%
% If you comment hyperref and then uncomment it, you should delete
% ReviewTempalte.aux before re-running LaTeX.
% (Or just hit 'q' on the first LaTeX run, let it finish, and you
%  should be clear).
\usepackage[pagebackref,breaklinks,colorlinks]{hyperref}


% Support for easy cross-referencing
\usepackage[capitalize]{cleveref}
\crefname{section}{Sec.}{Secs.}
\Crefname{section}{Section}{Sections}
\Crefname{table}{Table}{Tables}
\crefname{table}{Tab.}{Tabs.}

\begin{document}

%%%%%%%%% TÍTULO - POR FAVOR ACTUALIZAR
\title{Detección y Clasificación de Patógenos Foliares mediante Visión por Computador}

\author{Primer Autor\\
%Institución1\\
%Dirección de Institución1\\
{\tt\small primerautor@i1.org}
% Para un artículo cuyos autores pertenecen todos a la misma institución,
% omita las siguientes líneas hasta el cierre ``}''.
% Se pueden añadir autores y direcciones adicionales con ``\and'',
% igual que el segundo autor.
% Para ahorrar espacio, use el correo electrónico o la página web, no ambos
\and
Segundo Autor\\
%Institución2\\
%Primera línea de dirección de institución2\\
{\tt\small segundoautor@i2.org}
\and
Tercer Autor\\
{\tt\small tercerautor@i2.org}
\and
Cuarto Autor\\
{\tt\small cuartoautor@i2.org}
}
\maketitle

%%%%%%%%% RESUMEN
\begin{abstract}
   Breve resumen del trabajo desarrollado, así como los principales resultados y contribuciones. 
   
   
\end{abstract}

%%%%%%%%% CUERPO DEL TEXTO
\section{Introducción}
\label{sec:intro}

Aquí se describe el problema a resolver (¿qué queremos hacer?), la motivación (¿por qué es relevante hacerlo?), y los objetivos (¿qué objetivos específicos vamos a abordar para resolver el problema?).

Usamos el formato \LaTeX\ de la conferencia \href{https://en.wikipedia.org/wiki/Conference_on_Computer_Vision_and_Pattern_Recognition}{CVPR}. Este documento puede escribirse tanto en inglés como en español. 

Lo que sigue es una aproximación tentativa de las secciones que el documento debería tener. Si los estudiantes consideran que necesitan otras, así como subdividir las secciones en diferentes subsecciones, pueden hacerlo sin problema.

Los estudiantes pueden inspirarse en la página web de proyectos finales de \href{http://cs231n.stanford.edu/index.html}{cs231n}: \url{http://cs231n.stanford.edu/project.html}, donde se presentan y desarrollan diferentes proyectos. 

Este informe final completo puede tener de 6 a 8 páginas (ni más ni menos). 

\section{Fundamentos Teóricos}

Esta sección presenta los conceptos fundamentales necesarios para entender el trabajo.


\section{Trabajos Relacionados}

Presenta lo que se ha hecho en el campo previamente, y cuáles son los mejores métodos actualmente. Es muy importante, en general, no solo en esta sección, documentar adecuadamente la literatura relevante. Para ello, debe usar el archivo $.bib$, de la manera que muestro aquí: \cite{mesejo2016computer, lathuiliere2019comprehensive, vargas2023deep}

\section{Métodos}

Descripción detallada de los métodos utilizados y/o propuestos, y justificación clara de por qué se usan estos métodos y no otros.

\section{Experimentos}

Aquí se presentan los datos utilizados, el protocolo de validación experimental, las métricas usadas, los experimentos realizados, los resultados obtenidos y su discusión.

\subsection{Conjunto de Datos}

\section{Conclusiones}

Sección que presenta, brevemente y a modo de resumen, las principales conclusiones del trabajo realizado. También suele incluir posibles trabajos futuros. Es decir, cuáles son las líneas más prometedoras para continuar con este trabajo, así como posibles propuestas de mejora. IMPORTANTE: estas son las conclusiones científicas alcanzadas en el proyecto; ¡no tus conclusiones personales sobre el trabajo que has realizado!



%%%%%%%%% REFERENCES
{\small
\bibliographystyle{ieee_fullname}
\bibliography{egbib}
}

\end{document}
