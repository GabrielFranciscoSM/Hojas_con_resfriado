\section{Datasets}
\label{sec:datasets}

% 1. Introducción y Selección
En esta sección se detallan los conjuntos de datos utilizados para el 
entrenamiento y validación de los modelos propuestos. La correcta 
elección de estos datos es fundamental para el desempeño del 
sistema, dado que se busca integrar tareas de clasificación taxonómica 
y detección de patologías en un único pipeline.

\subsection{Criterios de Selección}
Para la conformación del dataset final, se han establecido una serie de 
requisitos estrictos derivados de la arquitectura del sistema (Clasificador + YOLO):

\begin{itemize}
    \item \textbf{Compatibilidad con Arquitectura en Dos Etapas:} Se requieren 
    datos que permitan entrenar tanto el modelo de clasificación de especies 
    (Etapa 1) como el detector de objetos (Etapa 2). Esto implica la necesidad de 
    imágenes con etiquetas a nivel de imagen (especie) y anotaciones a nivel de 
    región (bounding boxes para lesiones).
    
    \item \textbf{Volumen y Diversidad:} Para entrenar arquitecturas basadas en 
    Vision Transformers y CNNs profundas (como YOLO), es necesario un volumen de 
    datos suficiente (en el orden de miles de imágenes por clase) para evitar 
    el sobreajuste. Además, las imágenes deben presentar variabilidad en 
    condiciones de iluminación, fondo y estadios de la enfermedad para garantizar 
    la capacidad de generalización del modelo en entornos no controlados.
    
    \item \textbf{Calidad de las Anotaciones:} Es crítico que las anotaciones de 
    las lesiones sean precisas y consistentes, ya que el rendimiento de YOLO 
    depende directamente de la calidad de las cajas delimitadoras (ground truth) 
    durante el entrenamiento.
\end{itemize}

Dado que no existe un único dataset público que satisfaga plenamente todas estas 
necesidades para múltiples especies simultáneamente, se ha optado por la 
agregación de fuentes heterogéneas.

% 2. Descripción Técnica
\subsection{Descripción de los Datasets}
Basándonos en los criterios anteriores, se han seleccionado los siguientes 
datasets públicos. La Tabla~\ref{tab:datasets_summary} resume sus 
características principales.

\begin{table*}[t]
    \centering
    \caption{Resumen técnico de los datasets seleccionados.}
    \label{tab:datasets_summary}
    \begin{tabular}{lcccc}
        \toprule
        \textbf{Dataset} & \textbf{Imágenes} & \textbf{Especies} & \textbf{Clases (Enf.)} & \textbf{Anotación} \\
        \midrule
        Apple Scab (Roboflow) & 2,400 & 1 & 3 & Bounding Box \\
        Tomato Diseases \cite{tomatoes-ddzvv_dataset} & 1,500 & 1 & 5 & Bounding Box \\
        \bottomrule
    \end{tabular}
\end{table*}

% 3. Análisis Exploratorio (EDA)
\subsection{Análisis Exploratorio de Datos}
Para garantizar la robustez de los modelos, se ha realizado un análisis de la distribución de clases y la variabilidad de las muestras.

\subsubsection{Distribución de Clases}
Como se observa en la Figura~\ref{fig:class_dist}, existe un desbalanceo significativo en ciertas clases de enfermedades, lo cual ha motivado el uso de técnicas de aumento de datos.

% Placeholder para la figura de distribución
\begin{figure}[h]
    \centering
    % \includegraphics[width=0.8\linewidth]{images/class_distribution.png}
    \caption{Histograma de frecuencia de imágenes por clase de enfermedad.}
    \label{fig:class_dist}
\end{figure}

\subsubsection{Análisis de Bounding Boxes}
Para el modelo YOLO, es crítico analizar el tamaño relativo de las lesiones. La Figura~\ref{fig:bbox_analysis} muestra la distribución espacial y de tamaño de las cajas delimitadoras, lo que justifica la elección de los \textit{anchors} predeterminados.

% 4. Preprocesamiento
\subsection{Preprocesamiento y Particionamiento}
Las imágenes han sido redimensionadas a $640 \times 640$ píxeles para la etapa de detección y normalizadas utilizando la media y desviación estándar de ImageNet.

El conjunto de datos se ha dividido siguiendo un esquema estratificado:
\begin{itemize}
    \item \textbf{Entrenamiento (Train):} 70\%
    \item \textbf{Validación (Val):} 15\%
    \item \textbf{Prueba (Test):} 15\%
\end{itemize}

Para mitigar el sobreajuste, se aplicaron transformaciones aleatorias durante el entrenamiento, incluyendo rotaciones ($\pm 15^\circ$), volteo horizontal y variaciones de brillo/contraste.
